\documentclass[12pt,letterpaper]{article}
\usepackage[left=2.65cm,top=2.5cm,right=2.65cm,bottom=2.5cm]{geometry} 
\usepackage[english]{babel}
\usepackage[utf8x]{inputenc}
\usepackage{amsmath}
\usepackage{eqnarray}
\usepackage{mathtools}
\usepackage{amsmath, amsthm, amssymb}
\usepackage{amssymb} 
\usepackage[retainorgcmds]{IEEEtrantools}
\usepackage{booktabs,caption}
\usepackage[flushleft]{threeparttable}
\usepackage{graphicx}
\usepackage{tabularx}
\usepackage{subfig}
\usepackage{kpfonts}    % for nice fonts
\usepackage{microtype} 
\usepackage{booktabs}   % for nice tables
\usepackage{bm}         % for bold math
\usepackage{listings}   % for inserting code
\usepackage{verbatim}   % useful for program listings
\usepackage{color}  
%\usepackage[colorlinks=true]{hyperref}%%TABLE % use for hypertext
\usepackage[colorlinks = true,
            linkcolor = blue,
            urlcolor  = blue,
            citecolor = blue,
            anchorcolor = black]{hyperref}
\usepackage[colorinlistoftodos]{todonotes}
\usepackage{natbib}
\renewcommand{\baselinestretch}{1.25}
\setlength{\parindent}{0cm}
\setlength{\parskip}{\baselineskip} 



\begin{document}

%+Title
\title{}


\date{}
 \maketitle
%
%
%
%{{\bf{Keywords}}: Care homes, house prices, long-term care, England\\
%\bf{JEL}: R31, I12}
%
%\newpage
%\tableofcontents
%\newpage



%\newpage

\subsection*{Specifications}



The purpose of our analysis is to study empirically the effects of the house prices on the proportion of care homes in local
  long term care markets. The main analysis is based on regressions that follow the baseline 
specification:
 
 \begin{eqnarray}
\label{equation1}
      C_{it} =\beta X_{it} + \alpha P_{it} + \epsilon_{it}
 \end{eqnarray}

 where $C$ is proportion of care homes per 1000 population over 65 in a local authority $i$ in a time period $t$,
  $P_{it}$ is the average of the house prices and 
 and $\epsilon$ represents an error term that is identically and independtly distributed.  $X_{it}$
  represents a vector with different observable variables
 that characterize the composition of local long term care markets and that we use as main controls in some 
 of our specifications. 
 
 The choice of these controls is guided by elements that characterize the composition of the demand 
for long term care and its needs. Firstly, we include the proportion 
  of people older than 85 and proportion of people that receive the attendance 
  allowance\footnote{This benefit aims to support those people with physical disabilities in UK that live
   independently and might require residential care services otherwise. } as proxies of the level 
   of health dependency. Also, given the association between the financial needs and the funding
    support determined by the means-test, we incorporate the proportion of people that receive some
     sort of income support and the proportion of people that receive pension credits to reflect the payer 
     composition within the local population. These variables have been previously used in the literature for these purposes 
     \citep{darton2010slicing, forder2014}. Likewise, given that long term care is a labour intense activity,
      we add the proportion of females that claim for job seekers’ allowance in order to get a proxy for unemployment.
      
      In addition to the former, we also include in $X$ a measure of the Herfindahl–Hirschman Index (HHI) to control for
      the competition between care homes in the local market. In our case, the HHI is a measure of concentration that 
      reflects the squared shares of beds across all the providers in a local market. The values range from
       0 to 1 where higher values represent higher concentration and therefore less competition.  
 
The parameter of interest, $\alpha$,  may be interpreted as a causal effect
  of the house prices on the distribution of care homes, only if
         $P_i$ is exogenous so that $Cov(P_{i}$,$\epsilon_{i}) = 0 $. Yet, a potential element that can lead 
         to inconsistent estimations of $\alpha$ may be the presence of unobserved
 variables that confound the effect of the house prices on the proportion of care homes.

 For tackling with these problems we consider an instrumental variables approach and instrument the house prices with 
instruments referred to the variation of 
 restrictiveness in the planning regulations. The measure that we use is the 
 rate of refusal of major projects. It is normally used in the literature and 
 reflects the share of applications corresponding to projects that entail 10 or more 
 dwelling that are rejected by a local authority during a year. 
 
 A potential problem with this instrument is that is procyclical and this may 
 entail endogeneity concerns. In order to address them we use two identification 
 strategies based on the 
 variation in the rate of delay of projects before and after a planning reform aimed at speeding up the planning processes 
 and the share of local political power. The specific instruments that we use are the 
change in the delay rate before and after the reform and the share of Labour 
voters in the local authority. 

In addition to the planning regulations, there may be other drivers that entail 
restrictions in the supply of houses and thus may lead to increases in the house 
prices. Physical constraints may be an example of those and should be included in the 
estimation. We use the share of developed land to express the extent of physical 
constraints. A potential limitation referred to this variable is that 
  the availability (or scarcity) of this type of land can be the result of elements that 
  also affect the house prices and therefore may imply endogeneity. For 
  addressing this problem, the historic population density can be used as an 
  instrument for identifying the share of developable land since it may show the 
  early forms of agglomeration. I use the historic population density in 1911.

Considering these caveats specification for estimating the first stage fitted 
values of the house prices is expressed

  \begin{eqnarray}
\label{first stage}
    P_{it} = \delta Z_{it} + \beta \chi_{it} + \eta_{i}+ \psi_{i} + u_{it} 
 \end{eqnarray}

 where $Z$ refers to the variable associated with the planning regulation (e.g. the rate of refusal of major projects), $\chi$  
 to the variable referred to the physical constraint (e.g. share of developed land) and $\psi$ binary variables 
 for each planning authority. In addition to the specification developed by 
\cite{hilber2016supply}, we include and additional control $\eta$ corresponding to the 
share of Labour voters for each local authority in the last national election in June 
2015. 
 
\section{Tables}

\begin{table}[h]
\centering
\caption{First stage results, dependent variable house prices (log)}
\label{table first_stage}
\resizebox{\textwidth}{!}{% 
\begin{tabular}{@{}lccc@{}}
\toprule

                                    &  & {Average house price (log) } & &{}
 
                       & Refusal rate & {Change rate of delay } & {Share votes of Labour} & {Population density}\\ 
\hline
                & 1.222***   & -0.095           & -1.672***  &        \\
                                    & (0.28)     & (0.066)               & (0.328)      &        \\
\hline
Observations                        & 945                   & 945                    & 945    &               \\
R2                                           & 0.694               &        0.672         &   0.695   &                 \\
F (excluded instruments)     & 19.42***     &          2.07              &     46.32***       &           \\
\bottomrule
\end{tabular}}
\begin{tablenotes}
      \scriptsize
      \item {\it{Notes}}: All regressions include the following controls. Share of people 85+, 
      Share of people receiving Attendance Allowance, Share of people with pension credits, 
      Share of females claiming for Job Seekers Allowance, Share of adults with income 
      support, 
      Herfindahl-Hirschmann Index, share of Labour voters for 2015. All regressions include fixed effect controls at
      county level. Robust standard errors in parentheses. Standard errors are clustered at local planning 
      authority level. ***/**/*/$^{+}$ denote significance levels at 1\%, 5\%, 
      10\% and 15\%. Standard errors are presented in parentheses. 
    \end{tablenotes}
\end{table}


\newpage
\bibliographystyle{apalike}
\bibliography{ex1}
%remember to use \citep{} for citation
%
\newpage
\end{document}
