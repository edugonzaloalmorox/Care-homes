\documentclass[11pt,a4paper,]{article}
\usepackage[]{kpfonts}
\usepackage{setspace}
\setstretch{1.5}
\usepackage{amssymb,amsmath}
\usepackage{ifxetex,ifluatex}
\usepackage{fixltx2e} % provides \textsubscript
\ifnum 0\ifxetex 1\fi\ifluatex 1\fi=0 % if pdftex
  \usepackage[T1]{fontenc}
  \usepackage[utf8]{inputenc}
\else % if luatex or xelatex
  \ifxetex
    \usepackage{mathspec}
  \else
    \usepackage{fontspec}
  \fi
  \defaultfontfeatures{Ligatures=TeX,Scale=MatchLowercase}
\fi
% use upquote if available, for straight quotes in verbatim environments
\IfFileExists{upquote.sty}{\usepackage{upquote}}{}
% use microtype if available
\IfFileExists{microtype.sty}{%
\usepackage{microtype}
\UseMicrotypeSet[protrusion]{basicmath} % disable protrusion for tt fonts
}{}
\usepackage[margin=1in]{geometry}
\usepackage{hyperref}
\PassOptionsToPackage{usenames,dvipsnames}{color} % color is loaded by hyperref
\hypersetup{unicode=true,
            pdftitle={The effect of house prices on entry in the market of long term care: Evidence from England},
            pdfauthor={Eduardo Gonzalo-Almorox; Nils Braakmann; Volodymyr Bilotkach; John Wildman; (Newcastle University Business School); (Preliminary version. Please do not cite)},
            colorlinks=true,
            linkcolor=blue,
            citecolor=Blue,
            urlcolor=Blue,
            breaklinks=true}
\urlstyle{same}  % don't use monospace font for urls
\usepackage[style=authoryear-comp]{biblatex}

\addbibresource{ex.bib}
\usepackage{longtable,booktabs}
\usepackage{graphicx,grffile}
\makeatletter
\def\maxwidth{\ifdim\Gin@nat@width>\linewidth\linewidth\else\Gin@nat@width\fi}
\def\maxheight{\ifdim\Gin@nat@height>\textheight\textheight\else\Gin@nat@height\fi}
\makeatother
% Scale images if necessary, so that they will not overflow the page
% margins by default, and it is still possible to overwrite the defaults
% using explicit options in \includegraphics[width, height, ...]{}
\setkeys{Gin}{width=\maxwidth,height=\maxheight,keepaspectratio}
\IfFileExists{parskip.sty}{%
\usepackage{parskip}
}{% else
\setlength{\parindent}{0pt}
\setlength{\parskip}{6pt plus 2pt minus 1pt}
}
\setlength{\emergencystretch}{3em}  % prevent overfull lines
\providecommand{\tightlist}{%
  \setlength{\itemsep}{0pt}\setlength{\parskip}{0pt}}
\setcounter{secnumdepth}{5}

%%% Use protect on footnotes to avoid problems with footnotes in titles
\let\rmarkdownfootnote\footnote%
\def\footnote{\protect\rmarkdownfootnote}

%%% Change title format to be more compact
\usepackage{titling}

% Create subtitle command for use in maketitle
\newcommand{\subtitle}[1]{
  \posttitle{
    \begin{center}\large#1\end{center}
    }
}

\setlength{\droptitle}{-2em}
  \title{The effect of house prices on entry in the market of long term care:
Evidence from England}
  \pretitle{\vspace{\droptitle}\centering\huge}
  \posttitle{\par}
  \author{Eduardo Gonzalo-Almorox \\ Nils Braakmann \\ Volodymyr Bilotkach \\ John Wildman \\ (Newcastle University Business School) \\ (Preliminary version. Please do not cite)}
  \preauthor{\centering\large\emph}
  \postauthor{\par}
  \predate{\centering\large\emph}
  \postdate{\par}
  \date{March 2017}


\begin{document}
\maketitle

\begin{abstract}

This study investigates the effects of house prices in the English care homes market. High house prices, as experienced currently in England, may disincentive the entry in certain markets restricting the access to long term care services in these areas. Alternatively, these areas may also suppose business opportunity. We provide evidence in order to disentangle these effects. Our results suggest that higher house prices increase the rate care homes. Based on unique dataset that collates information from several sources our analysis exploits planning regulations to address empirical limitations associated with the house prices. Our findings contribute to inform policy makers about the relationship between the long term care and housing markets. 

 
\end{abstract}

\textbf{JEL Codes}: I11, L22, R31, C26

\newpage

\section{Introduction}\label{introduction}

England has experienced the fastest growth in house prices amongst all
OECD country during the last decades. This inflationary trend has had
consequences for both households, materialised in the so called ``house
affordability crisis'', and to less extent businesses. In this paper we
investigate the relationship between the house prices and the market
structure of an industry that typically operates with low margins, the
care homes that provide long term care services. Our interest in the
long term care is not trivial. Elements such as the ageing of the
population or some socioeconomic changes that include the inclusion of
more women in the labour force as well as the composition of different
family structures, have shifted informal caregiving towards more formal
long term care provision. These patterns evidence the importance of this
sector in the forthcoming decades. Yet, despite the will of policy
makers to design policies that preserve a sustainable provision of long
term care and that also ensure competitive market structures, there is
limited evidence for the design of these policies. We aim at informing
these policies by analysing the extent of the effect of high prices in
the housing market on the entries in the market of care homes.

A major characteristic of the English market for long term care services
consists of the geographical disparities in the levels of provision and
funding and the consequent effects on other variables of interest for
the market structure. Forder and Fernández (2012) analysing data at
local level, highlight the substantial differences in both the level of
need and the unit costs across English councils. This spatial
variability seems to appear also in other variables of interest such as
the levels of social care expenditure (Fernández and Forder (2015)).
According to these authors, a principal element to explain this local
divergences are the underlying conditions that providers face for
supplying their services. Considering this, we argue that an element
that may influence the structure of the market for long term care
services concerns the situation relative to the housing market. Alike
the former, the housing market in England also presents notable
divergences across local areas that lead to different levels of prices.
Several authors have argued that this is a direct consequence of the
different designs associated with the planning regulations across the
country. Hence, more restrictive planning regulations result in higher
levels of house prices (see for instance, Cheshire (2009) or Hilber and
Vermeulen (2016) for comprehensive reviews).

The effect of house prices on the market entry of care homes is a priori
uncertain. One possible explanation may consist of the influence of
house prices as a cost for running a care home. Hence, high house prices
may suppose an important barrier that can restrict the entry in certain
markets. A potential consequence derived from the former, people living
in these areas may find less long term care choices closer to them. A
second argument may be on the basis of how high house prices may
represent a business opportunity. The segment of the population that
benefits from current upward trend in the house prices are those elderly
homeowners that are able to monetize the higher value of their asset by
selling their houses and moving out cheaper areas (Hilber and Schöni,
2016). If this argument holds, areas with higher prices may be
associated with greater levels of affluence and consequently greater
proportions of clients that are more willing to pay for the services of
a care home. Although the latter may contribute to preserve the
financial viability of care homes in the market, an issue that
constitutes a current public policy concern, it may also result in an
unequal distribution of long term care across different areas in England
where the most affluent areas are more benefited from a greater supply
of home care services.

In order to proceed with our analysis, we construct a unique dataset
that merges information from several sources to collate information
regarding the characteristics of the dynamics in the care homes market,
the housing markets and the planning regulations. The dataset captures
information regarding local authorities at different level (e.g.~street,
district and county level). A first technical hurdle concerning the
dataset, consists of distinguishing de novo entries associated with
providers that effectively produce a new activity. Secondly, there is an
additional empirical caveat that we have to address with regards to
effect of house price on care homes entries. It may be possible that
care homes select markets that have high prices on a non-random basis
due to unobservable variables. This sample selection bias may invalidate
the estimates corresponding to the effects of house prices. In order to
overcome these, we carry out and identification strategy which uses an
instrumental variables approach that exploits the variability in the
restrictiveness of planning regulations across English districts. Our
identification relies on the assumption that changes in the planning
requirements affect the entry of care homes in market through the levels
of house prices. Our instrumental variables estimates indicate a
positive relationship between the market entries and the house prices
suggesting increases in the proportion of care homes of 0.21 when house
prices increase by a 10\%. We then can argue that higher house prices
lead to greater market entries. These results suggest that providers
would be focusing on areas where there are more expensive houses.

To the best of our knowledge, no previous studies have been undertaken
to provide causal evidence with regards to the effects of housing prices
in the context of entries in the market of care homes. This research
also makes a number of contributions to several strands of the
literature. It provides further evidence to the growing literature that
analyses aspects associated with the market of care homes in England.
Forder and Allan (2014) study the elements that determine the
competition amongst care homes and assess the consequences of this
competition in both prices and quality. Also Allan and Forder (2015)
evaluate empirically the causes of market exits by investigating the
effects of maintaining minimum standards in the quality of the service.
We extend this literature by addressing issues referred to the entry of
care homes in the market. Prior to this paper, only Machin et al (2003)
have provided empirical evidence of factors affecting the market entry
by analysing the effects of setting of a minimum wage. In addition to
providing a more up to date evidence, this research uses a more
extensive dataset provided by the regulator, the Care Quality Commission
(CQC). Likewise, this research also extends the literature that studies
the effects of the planning system and the high house prices in England
using the care homes as a new sector for the analysis.

The rest of the paper is organised as follows. In the next section we
introduce the institutional framework corresponding to the organisation
of the local authorities responsible for planning activities and long
term care. In Section 3 we outline the details associated with the
sources of data and the variables used. Section 4 describes the
empirical model and section 5 present the results. Finally, the main
conclusions of the paper are shown in section 6.

\section{Institutional background}\label{institutional-background}

In England planning and long term care are activities that are ruled and
applied by local governments. The structure of these is nonetheless
complex and entails different organizational levels, depending on the
type of services that are regulated. In this section we outline the main
characteristics of the local government in England considering the
particular cases of planning regulations and long term care. This will
help to understand the geography that we adopting for our empirical
analysis.

\subsection{The English market for long term
care}\label{the-english-market-for-long-term-care}

\printbibliography


\end{document}
