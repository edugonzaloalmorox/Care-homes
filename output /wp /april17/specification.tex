\documentclass[12pt,letterpaper]{article}
\usepackage[left=2.65cm,top=2.5cm,right=2.65cm,bottom=2.5cm]{geometry} 
\usepackage[english]{babel}
\usepackage[utf8x]{inputenc}
\usepackage{amsmath}
\usepackage{eqnarray}
\usepackage{mathtools}
\usepackage{amsmath, amsthm, amssymb}
\usepackage{amssymb} 
\usepackage[retainorgcmds]{IEEEtrantools}
\usepackage{booktabs,caption}
\usepackage[flushleft]{threeparttable}
\usepackage{graphicx}
\usepackage{tabularx}
\usepackage{subfig}
\usepackage{kpfonts}    % for nice fonts
\usepackage{microtype} 
\usepackage{booktabs}   % for nice tables
\usepackage{bm}         % for bold math
\usepackage{listings}   % for inserting code
\usepackage{verbatim}   % useful for program listings
\usepackage{color}  
%\usepackage[colorlinks=true]{hyperref}%%TABLE % use for hypertext
\usepackage[colorlinks = true,
            linkcolor = black,
            urlcolor  = black,
            citecolor = black,
            anchorcolor = black]{hyperref}
\usepackage[colorinlistoftodos]{todonotes}
\usepackage{natbib}
\renewcommand{\baselinestretch}{1.25}
\setlength{\parindent}{0cm}
\setlength{\parskip}{\baselineskip} 



\begin{document}

%+Title
\title{}


\date{}
 \maketitle
%
%
%
%{{\bf{Keywords}}: Care homes, house prices, long-term care, England\\
%\bf{JEL}: R31, I12}
%
%\newpage
%\tableofcontents
%\newpage



%\newpage

\subsection*{Specifications}



The purpose of our analysis is to study empirically the effects of the house prices on the proportion of care homes in local
  long term care markets. The main analysis is based on regressions that follow the baseline 
specification:
 
 \begin{eqnarray}
\label{equation: carehomes_prices}
      C_{it} =\beta X_{it} + \alpha P_{it} + \epsilon_{it}
 \end{eqnarray}

 where $C$ is proportion of care homes per 1000 population over 65 in a local authority $i$ in a time period $t$,
  $P_{it}$ is the average of the house prices and 
 and $\epsilon$ represents an error term that is identically and independtly distributed.  $X_{it}$
  represents a vector with different observable variables
 that characterize the composition of local long term care markets and that we use as main controls in some 
 of our specifications. 
 
 The choice of these controls is guided by elements that characterize the composition of the demand 
for long term care and its needs. Firstly, we include the proportion 
  of people older than 85 and proportion of people that receive the attendance 
  allowance\footnote{This benefit aims to support those people with physical disabilities in UK that live
   independently and might require residential care services otherwise. } as proxies of the level 
   of health dependency. Also, given the association between the financial needs and the funding
    support determined by the means-test, we incorporate the proportion of people that receive some
     sort of income support and the proportion of people that receive pension credits to reflect the payer 
     composition within the local population. These variables have been previously used in the literature for these purposes 
     \citep{darton2010slicing, forder2014}. Likewise, given that long term care is a labour intense 
     activity and it is specially carried out by females\footnote{This proportion is about 82\% in 2016 \citep{workforce2016}}
      we add the proportion of females that claim for job seekers’ allowance in order to get a proxy for 
      unemployment. In addition to the former, we also include in $X$ a measure of the Herfindahl–Hirschman Index (HHI) to control for
      the competition between care homes in the local market. In our case, the HHI is a measure of concentration that 
      reflects the squared shares of beds across all the providers in a local market. The values range from
       0 to 1 where higher values represent higher concentration and therefore less competition.  
 
Equation \ref{equation1} can be estimated by OLS and the parameter of interest, $\alpha$,  may be interpreted as a causal effect
  of the house prices on the distribution of care homes, only if
         $P_i$ is exogenous so that $Cov(P_{i}$,$\epsilon_{i}) = 0 $. 
         Nonetheless, the entry of a care home in a particular local long term 
         care market may be determined by additional factors other than house 
         prices that scape to the control of observable variables. If the 
         influence of these potential unobservable shocks is not appropriately 
         undertaken, the OLS estimations of $\alpha$ may be inconsistent. 
         
  An example that may illustrate the latter could be an unobserved shock that may affect positively the values of the properties and also incentivise
   the entries in the market given likely wealth effects. Hence, higher level of housing prices may result in
    wealth effects that lead to greater levels of consumption and then attract businesses. This implies that the selection
     of an area by a care home provider is likely to be \textit{known} -non-random,  and the effect of $P_{i}$ may be associated partially 
     with $\epsilon$. Likewise another potential problem may arise on the basis 
     that, in addition of $C_{i}$ being determined by $P_{i}$, $P_{i}$ also could be determined partially as 
     a function of $C_{i}$
     
     In order to tackle with these problems associated with $P_{it}$, we consider an instrumental variables (IV) approach, 
     where we use an instrumental variable $z$  that is uncorrelated with $\epsilon$ but is correlated
       with $P_{i}$. Inspired by \citet{hilber2016supply} our identification 
       strategy exploits the restrictiveness in the local planning regulations. 
       \citet{hilber2016supply} use this variability for analysing the effects of local earnings
 on house prices. Their findings confirm the vision that tight supply regimes – e.g. with more regulatory 
 constraints in the planning regulations, lead to increases in the prices. In our case, however, we apply 
the planning regulation variable as direct instrument to the house prices. For our identification we assume that this instrument,
  in addition to being correlated with the local earnings, is also correlated with the house prices. The measure that we use is the 
 rate of refusal of major projects. It is normally used in the literature and 
 reflects the share of applications corresponding to projects that entail 10 or more 
 dwellings that are rejected by a local authority during a year.  
 
 Both the relationship between
  planning regulations and house prices as well as the use of planning regulations 
  for addressing endogeneity bias associated with house prices have been well documented in the literature. 
  Considering the case of UK for instance, several authors have shed light with regards to the effects of tight planning 
  regulations on house prices suggesting a positive relationship \citep{cheshire2009, cheshire2014, barker2004barker, hilber2016supply}.
  A potential problem with this instrument is that is procyclical and this may 
be introduce bias. For example, developers may modify the way they apply when they are aware of the level
 of tightness of certain local planning authorities are tighter than others. It may happen that if they know
  that some local planning authorities are particularly restrictive they may withdraw their applications and focus
   on other markets. If this occurs, then the observed refusal rates may not reflect the level of real 
   restrictiveness, especially in the cases of more limiting local planning authorities. 
 For coping with this limitation it is possible to exploit two identification strategies on the basis of \citet{hilber2016supply}.
    
    The first involves a planning reform aimed at speeding up the planning processes and the
   second links the planning regulations and the variation in the share of local political power. 
The main idea corresponding to the identification strategy based on the planning reform consists
 of exploiting the variation in the change in the delay rates before and after the reform. Set in 2002,
  the reform included the establishment of an explicit goal for major development projects. The main
   purpose of this target was to avoid the delays of major projects by local planning authorities. 
   Even though they were not formally penalised for not meeting the target, local planning authorities did not have
     the incentive for neglecting the target either. The central government could retain financial resources
      addressed to local planning authorities. An option for local authorities to meet the target was to refuse 
      greater projects and conversely approve smaller projects which could be finished on time. 
      
On the basis of the former, it is possible to think on the behaviour of the local planning authorities
 before and after the reform paying particular attention to their level of restrictiveness. Thus, before 
 the reform local planning authorities that were more restrictive would be also the ones that had greater 
 delays and thereby the least likely to meet the target. Once the reform was established, these local 
 planning authorities would be also the ones more likely to refuse more projects and therefore suffer less
  delays. Less restrictive local planning would not have to alter their behaviour substantially.
   Considering this, we allow for a 10-year period to represent the average delay rates pre and post reform.
    Hence we consider the delay rates 1994 and 1996 and the delay rates between 2004-2006.  
    
The alternative strategy consists of taking advantage of the relationship between the political composition of local councils
 and the application of local planning regulations. In addition to 
 \citet{hilber2016supply},
  similar strategies have been used by other authors such \cite{bertrand2002does} or 
  \cite{sadun2015}. 
  Hence, we use the share of Labour party votes at the General Election of 1983.
   The information is obtained from the British Election Studies Information System.  
   We choose the share of Labour voters since the attitudes of these voters 
   regarding construction
   will be more on the basis of the job implications and inclined to grant house access rather than to preserve
    the value of the properties \citep{cheshire2016}.
    Also, we could have used the results derived from local elections. 
    Yet, these might be correlated with the development of local housing markets and constitute a source
     of potential bias. The time frame of 1983 provides the earliest date where election results can be linked
      to data corresponding to local authorities and then minimizes the potential association between the
       outcome of the election and the planning process.



In addition to the planning regulations, there may be other drivers that entail 
restrictions in the supply of houses and thus may lead to increases in the house 
prices. Physical constraints may be an example of those and should be included in the 
estimation. We use the share of developed land to express the extent of physical 
constraints. A potential limitation referred to this variable is that 
  the availability (or scarcity) of this type of land can be the result of elements that 
  also affect the house prices and therefore may imply endogeneity. For 
  addressing this problem, the historic population density can be used as an 
  instrument for identifying the share of developable land since it may show the 
  early forms of agglomeration. We use the historic population density in 1911 
  as an instrument for the share of developable land. 
  
Considering these caveats, we specify equation (\ref{equation: firststage_prices}) in order to estimate first-stage fitted values of the
 house prices. The predicted values derived from this equation are used then as instruments and incorporated in
 (\ref{equation: carehomes_prices}) in order to get a consistent estimate of $\alpha$

  \begin{eqnarray}
\label{equation: firststage_prices}
    P_{it} = \delta Z_{it} + \beta \chi_{it} + \eta_{i} + u_{it} 
 \end{eqnarray}

 where $Z$ refers to the variable associated with the planning regulation (e.g. the rate of refusal of major projects) and $\chi$  
 to the variable referred to the physical constraint (e.g. share of developed land). In addition to the specification developed by 
\cite{hilber2016supply}, we include $\eta$ as a control contemporaneous share 
share of Labour voters for each local authority corresponding the national election of June 
2015.  As we introduced before a potential issue for the reluctance of this instrument may be related to 
unobserved trends. For instance, some areas have been exposed to the inflow of
certain residents that may changed the demographic composition of certain areas and this also modified the voting behaviour. In order to 
control for this we include the share vote for each local authority 
corresponding to the results of the last national elections celebrated in June 
2015.\footnote{ \citet{cheshire2016} use this instrument for analysing the effect of planning regulations on the 
proportion of vacant houses in England. They provide housing markets in Greater London as an example of areas
that could have changed  their voting behaviour as a consequence of these inflows.}

Furthermore, another difference of our approach in comparison to their empirical strategy from \citet{hilber2016supply}, is that we use two instruments for 
identifying the house prices rather than a single instrument for identifying variables that determine them such as the 
planning regulations and the share of developed land respectively.
 
\section{Tables}

\begin{table}[h]
\centering
\caption{First stage results, dependent variable house prices (log)}
\label{table first_stage}
\resizebox{\textwidth}{!}{% 
\begin{tabular}{@{}lccc@{}}
\toprule

                                    &  & {Average house price (log) } & &{}
 
                       & Refusal rate & {Change rate of delay } & {Share votes of Labour} & {Population density}\\ 
\hline
                & 1.222***   & -0.095           & -1.672***  &        \\
                                    & (0.28)     & (0.066)               & (0.328)      &        \\
\hline
Observations                        & 945                   & 945                    & 945    &               \\
R2                                           & 0.694               &        0.672         &   0.695   &                 \\
F (excluded instruments)     & 19.42***     &          2.07              &     46.32***       &           \\
\bottomrule
\end{tabular}}
\begin{tablenotes}
      \scriptsize
      \item {\it{Notes}}: All regressions include the following controls. Share of people 85+, 
      Share of people receiving Attendance Allowance, Share of people with pension credits, 
      Share of females claiming for Job Seekers Allowance, Share of adults with income 
      support, 
      Herfindahl-Hirschmann Index, share of Labour voters for 2015. All regressions include fixed effect controls at
      county level. Robust standard errors in parentheses. Standard errors are clustered at local planning 
      authority level. ***/**/*/$^{+}$ denote significance levels at 1\%, 5\%, 
      10\% and 15\%. Standard errors are presented in parentheses. 
    \end{tablenotes}
\end{table}


\newpage
\bibliographystyle{apalike}
\bibliography{ex1}
%remember to use \citep{} for citation
%
\newpage
\end{document}
